\begin{enumerate}
  \item Summary \(40\% : 2.5 pages\)
    \begin{enumerate}[label*=\arabic*.]
      \item Motivation \(8\%\)
      \item Contribution \(8\%\)
      \item Methodology \(16\%\)
      \item Conclusion \(8\%\)
    \end{enumerate}
  \item Critique \(30\% : 1.5 pages\)
    \begin{enumerate}[label*=\arabic*.]
      \item Title of 1st Critique \(15\%\)
       \item Title of 2nd Critique \(15\%\)
       \item Optional: Title of 3rd Critique
    \end{enumerate}
  \item Synthesis \(30\% : 1 page\)
    \begin{enumerate}[label*=\arabic*.]
      \item Title of 1st Idea \(30\%\)
      \item Optional: Title of 2nd Idea
    \end{enumerate}
\end{enumerate}
1. What is the research problem the paper attempts to address?
O artigo contextualiza o cenário atual no que diz respeito à manipulação de imagens 
por ferramentas de edição, ressaltando a ampla gama de técnicas que essas ferramentas
possuem e que permitem aos seus usuários a manipulação de imagens que implica em uma análise
detalhista por um profissional especializado para descobrir algum tipo de fraude.

What is the motivation of the research work?
Não obstante, a motivação principal do trabalho proposto é de criar uma ferramenta capaz de
avaliar uma dada imagem, de forma que qualquer pessoa comum possa identificar potenciais fraudes,
sem a necessidade de auxílio de um analista forense, por exemplo.
Podemos destacar, ainda, o fato de que pessoas podem coletar evidencias de crimes ou 
eventos quaisquer de uma forma trivial considerando a difusão de dispositivos móveis 
equipados com cameras de boa qualidade.

Is there a crisis in the research field that the paper attempts to resolve?
Podemos afirmar que ainda não há crise na área pesquisada, porém, tendencias de mercado
norteiam para que processos historicamente feitos sob arquivos impressos e com a presença
dos envolvidos, sejam totalmente realizados por plataformas online, o que pode, futuramente,
causar grandes transtornos às empresas e ao estado de um modo geral pelo risco de desvio de conduta
e possibilidades de fraudes em processos sigilosos, de grande valor agregado, avaliações contratuais, etc.
Não é demasiado lembrar que processos judiciais já são interpretados com ajuda de imagens digitais, o que 
nos remete a necessidade de garantir a autenticidade das mesmas, assim como também no jornalismo, 
cuja integridade pode ser colocada em xeque em casos de adulteração de fotos em quaisquer publicações.

Is the research work attempting to overcome the weaknesses of existing approaches?
Is an existing research paradigm challenged? In short, what is the niche of the paper?

2. What are the claimed contributions of the paper? What is new in this paper?
A new question is asked?
A new understanding of the research problem?
A new methodology for solving problems?
A new algorithm?
A new breed of software tools or systems?
A new experimental method?
A new proof technique?
A new formalism or notation?
A new evidence to substantiate or disprove a previously published claim?
A new research area?
In short, what is innovative about this paper?

3. How do the authors substantiate their claims?
What is the methodology adopted to substantiate the claims?
What is the argument of the paper? What are the major theorems?
What experiments are conducted?
Data analyses?
Simulations?
Benchmarks?
User studies?
Case studies?
Examples?
In short, what makes the claims scientific (as opposed to being mere opinions1)?

4. What are the conclusions?
What have we learned from the paper?
Shall the standard practice of the field be changed as a result of the new findings?
Is the result generalizable?
Can the result be applied to other areas of the field?
What are the open problems?
In short, what are the lessons one can learn from the paper?

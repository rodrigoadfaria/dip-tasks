\documentclass[12pt]{article}
\usepackage[latin1,utf8]{inputenc}
\usepackage[brazil]{babel}
\usepackage{setspace}
\usepackage{boxedminipage}
\usepackage{latexsym}
\usepackage{multirow}
\usepackage[pdftex]{graphicx}
\usepackage{float}
\usepackage{url}
\usepackage{xcolor,listings}
\usepackage{amsmath,mathrsfs}
\usepackage{biblatex}
\usepackage[portuguese, ruled, linesnumbered]{algorithm2e}
\usepackage{enumitem}
\usepackage[export]{adjustbox}
\addbibresource{references.bib}

%\setlength{\parskip}{0.1cm}
\setlength{\paperheight}{29.7cm}
\setlength{\textheight}{23.0cm}
\setlength{\textwidth}{16.5cm}
\setlength{\oddsidemargin}{0.0cm}
\setlength{\topmargin}{-1.0cm}
\pagestyle{empty}

% Custom colors
\usepackage{color}
\definecolor{deepblue}{rgb}{0,0,0.5}
\definecolor{deepred}{rgb}{0.6,0,0}
\definecolor{deepgreen}{rgb}{0,0.5,0}

\begin{document}

\lstset{language=python,
    keywordstyle=\color{deepblue}\bfseries,
    commentstyle=\color{deepgreen},
    stringstyle=\ttfamily\color{deepred!50!brown},
    breaklines=true,
    showstringspaces=false}
\lstset{literate=%
   *{0}{{{\color{red!20!violet}0}}}1
    {1}{{{\color{red!20!violet}1}}}1
    {2}{{{\color{red!20!violet}2}}}1
    {3}{{{\color{red!20!violet}3}}}1
    {4}{{{\color{red!20!violet}4}}}1
    {5}{{{\color{red!20!violet}5}}}1
    {6}{{{\color{red!20!violet}6}}}1
    {7}{{{\color{red!20!violet}7}}}1
    {8}{{{\color{red!20!violet}8}}}1
    {9}{{{\color{red!20!violet}9}}}1
}

\begin{center}
{\sf {\large Visão e Processamento de Imagens - Avaliação única -
    Parte II}}

\textbf{Preste atenção para as regras da prova}
\end{center}
1- O fonte latex (.tex) da prova será disponibilizado para facilitar
que você não tenha que copiar o enunciado das questões. Todas as questões
devem ser respondidas no mesmo arquivo.

\noindent 2- A prova é \textbf{individual}.  É permitido a consulta a
livros, apontamentos ou Internet, desde que devidamente referenciada.
Não é permitida a consulta a colegas, amigos, família, cachorro,
papagaio e etc. 

\noindent 3- A prova deve ser entregue diretamente no Paca, assim como
todos os códigos e imagens devem ser entregues no mesmo arquivo
comprimido.  \textbf{Duração da prova: 14 dias}.  

\noindent 4- Cada questão vale 20 pontos (pois são apenas 3 questões)
para a graduação e 15 pontos para a pós-graduação (pois são 4 questões). 
\bigskip

\begin{itemize}
\item[{\bf Q1.}] Para fazer esta questão, leia primeiro o artigo abaixo:
\begin{itemize}
\item \url{http://www.eecs.berkeley.edu/Pubs/TechRpts/2015/EECS-2015-85.pdf}
\end{itemize} 
\begin{itemize}
\item Faça um resumo do artigo de acordo com as indicações que deixei no paca (artigos sobre como fazer um resumo).

\begin{enumerate}
  \item \textbf{Resumo}
    \begin{enumerate}[label*=\arabic*.]
      \item \textbf{Motivação}

O artigo contextualiza o cenário atual no que diz respeito à manipulação de imagens por ferramentas de edição, ressaltando a ampla gama de técnicas que essas ferramentas possuem e que permitem aos seus usuários a manipulação de imagens com extrema facilidade, o que implica em uma análise detalhista por um profissional especializado para descobrir qualquer tipo de fraude.
\\[6pt]
Aliado a este fator, pode-se destacar ainda que pessoas podem coletar evidências de crimes ou eventos quaisquer de uma forma trivial, considerando a difusão de dispositivos móveis equipados com câmeras de boa qualidade em qualquer lugar do mundo.
\\[6pt]
Podemos afirmar que ainda não há crise na área pesquisada, porém, tendências de mercado norteiam para que processos historicamente feitos sob arquivos impressos e com a presença dos envolvidos, sejam totalmente realizados por plataformas online, o que pode, futuramente, causar grandes transtornos às empresas e ao estado de um modo geral pelo risco de desvio de conduta e possibilidades de fraudes em processos sigilosos, de grande valor agregado, avaliações contratuais, etc.
\\[6pt]
Não é demasiado lembrar que processos judiciais já são interpretados com ajuda de imagens digitais, o que nos remete a necessidade de garantir a autenticidade das mesmas, assim como também no jornalismo, cuja integridade pode ser colocada em xeque em casos de adulteração de fotos em quaisquer publicações.
\\[6pt]
Sendo assim, a motivação principal do trabalho proposto é de criar uma ferramenta capaz de avaliar uma dada imagem, de forma que qualquer pessoa comum, entenda-se por aquela que não tem conhecimento específico na área de processamento de imagens, possa identificar potenciais alterações, sem a necessidade de auxílio de um analista forense, por exemplo.
\\[6pt]
Tal ferramenta deve ser uma plataforma web, disponível em larga escala, sendo que uma das principais técnicas abordadas pelos autores para compor a paleta de funcionalidades do serviço é o Error Level Analysis (ELA).
\\[6pt]
É sabido que imagens JPEG perdem informação durante a compactação e uma certa quantidade de erro é introduzida cada vez que uma imagem JPEG é comprimida / salva novamente \cite{krawetz}. Logo, cada pixel contém um certo nível de erro, e a execução de uma nova compressão irá alterar o nível de erro do mesmo. No entanto, se uma imagem é manipulada, é muito provável que as diferentes partes da imagem terão diferentes níveis de erro, e o ELA é uma técnica que se concentra neste artefato para identificar se uma imagem foi manipulada e encontrar regiões adulteradas \cite{krawetz}.
\\[6pt]
\item \textbf{Contribuição}

Do ponto de vista de mercado, o trabalho apresentado visa um nicho não explorado pelos concorrentes avaliados, já que fornece um serviço com resultados de detecção mais informativo do que outras plataformas disponíveis.
\\[6pt]
Em outras palavras, os concorrentes focam apenas em análises de baixo nível das imagens e o serviço proposto, além das funcionalidades encontradas nos concorrentes avaliados, também adiciona suporte a outras avaliações de alto nível, tais como, detecção de regiões na imagem que foram copiadas/coladas.
\\[6pt]
Por outro lado, do ponto de vista técnico e científico, o autor ressalta que o ELA tem uma desvantagem: a região de alta frequência na imagem dada afetará o resultado, já que a compressão JPEG na imagem, naturalmente, comprime mais as regiões de alta frequência e, sendo assim, a parte de alta frequência de uma imagem terá, geralmente, uma maior queda nos níveis de erro, mesmo que não seja adulterada.
\\[6pt]
O autor propõe, então, superar esta deficiência do método criando uma máscara de baixa frequência que é gerada pela execução de um filtro gaussiano na imagem e medindo a diferença entre a imagem dada e a imagem filtrada, já que as regiões de baixa frequência terão alterações menores após filtro o gaussiano. 
\\[6pt]
Em seguida, esta máscara é multiplicada em cada um dos três canais diferentes da imagem resultante do ELA, o que auxilia na percepção visual das diferenças. Além disso, regiões de baixa frequência serão destacadas para demonstrar regiões com alto risco de terem sofrido alterações.
\\[6pt]
Outras técnicas também foram apresentadas, com o intuito de integração na plataforma web que está sob desenvolvimento pela equipe em questão. Considerando o processo de codificação de uma imagem, deu-se foco na etapa de quantização, já que nesta etapa, o processo de codificação pode deixar rastros que permitem descobrir se uma imagem foi manipulada ou não. Uma delas é o \textit{blocking artifact measurement} e a outra é o próprio ELA, com um realce no aspecto visual, no intuito de melhorar a experiência proporcionada pelo método original.
\\[6pt]
Uma outra técnica citada foi a análise de metadados, que consiste em avaliar algumas informações tais como, a ferramenta que foi utilizada para criar a imagem e, no caso de uma possível alteração, a ferramenta que alterou a imagem, que ficam gravadas no arquivo. Quando da alteração por um \textit{smartphone}, o sistema operacional também fica armazenado. Outra informação importante é data de criação e/ou alteração da imagem. Apesar de serem características que podem ser um indicativo de que a imagem foi alterada, nada garante que terceiros tentem burlar, também, o conteúdo relativo ao metadado da imagem.
\\[6pt]
\item \textbf{Metodologia}

O método então utilizado pelo autor foi calcular a norma L2 dos dados da imagem resultante da aplicação do ELA. Em seguida, um \textit{threshold} é aplicado na mesma imagem. Logo após este passo, o resultado do \textit{threshold} do ELA é submetido a um filtro gaussiano para, então, produzir uma região contínua. Finalmente, a imagem resultante é suavizada e submetida a um novo \textit{threshold} para remover o ruído.
\\[6pt]
Todo este processo cria regiões rotuladas de baixa frequência e regiões com alto índice de ELA tornam-se visualmente mais contundentes, quando da avaliação de qualquer tipo de violação da imagem.
\\[6pt]
De certa forma um novo algoritmo foi implementado, mas não foi apresentado pelos autores, apenas os passos que foram tomados para que pudessem ser gerados os filtros gaussianos, bem como sua aplicação na imagem.
Apesar do método ter sido citado, não há teoremas detalhados no artigo, apenas algumas técnicas, além do ELA, foram citadas, de forma que a ideia principal pudesse ser passada aos leitores. Além disso, o artigo tem um viés bem mais comercial do que técnico-científico, o que o torna muito parecido com um plano de negócio.
\\[6pt]
\item \textbf{Conclusão}

Pode-se destacar que o foco principal do artigo é o ELA, onde a contribuição de um dos autores, Yan Zhao, fica mais evidente, tanto que uma seção foi inteiramente dedicada à sua colaboração no trabalho.
\\[6pt]
Dos experimentos realizados, uma imagem foi dada como exemplo para mostrar a técnica com \textit{blocking artifact measurement}. Com o ELA, 3 \textit{datasets} com imagens manipuladas foram utilizados: \textit{CASIA2 Tampered Image Detection Evaluation Database} que consiste de 7491 imagens autênticas e 5123 adulteradas, o segundo \textit{Columbia Uncompressed Image Splicing Detection Evaluation Dataset} formado por 180 imagens remontadas, fruto de cópia de um recorte da própria imagem ou de uma outra fonte, transposta para outra região e 180 imagens originais e o terceiro é o \textit{Benchmark Dataset} composto, também, por imagens de alta qualidade com regiões adulteradas por cópia e cola.
\\[6pt]
O resultado do método ELA original elaborado por Krawetz \cite{krawetz} foi exibido e, posteriormente, a aplicação do novo método proposto com filtros gaussianos mostram resultados significativos.
\\[6pt]
Além do aspecto visual, Yan Zhao também criou um classificador capaz de decidir se uma imagem foi modificada ou não. A extração dos atributos foi feita com dados numéricos do ELA como a média, a mediana e a variância nos três canais da imagem resultante do ELA. O classificador foi criado com \textit{support vector machine} e \textit{adaptive boosting}, ambos do pacote \textit{Scikitlearn}.
\\[6pt]
Os resultados obtidos com o classificador foram relevantes, sendo uma taxa de acerto de 84.34\% com \textit{support vector machine} e 85.54\% com \textit{adaptive boosting} e árvores de decisão.
\\[6pt]
Apesar dos resultados terem sido expressivos, o experimento com o \textit{dataset} \textit{Benchmark Dataset} não foi bem sucedido, conforme citado pelos autores, mas nenhuma métrica ou medida de erro foi referenciada.
\\[6pt]
Um \textit{benchmark} também foi elaborado com duas empresas diferentes. A primeira, System of Methods and Tools of Digital Processing Technology LLC, conhecida como SMTDP, é uma empresa de tecnologia russa fundada em 2011 que se concentra em automação de processos de negócios e detecções de manipulação de imagens. Algumas das tecnologias usadas pela SMTDP são a análise de metadados de imagem e análise de compressão. A SMTDP concentra-se em parcerias com outras grandes empresas como a Belkasoft e PricewaterhouseCoopers.
\\[6pt]
A segunda empresa é chamada de Verifeyed e está localizada na República Checa. Tanto as ferramentas, quanto o mercado consumidor são diferentes se comparadas com a SMTDP. A Veryfied tem um pacote de software, que é comercializado sob a venda de licenças diretamente aos usuários finais. A tecnologia principal concentra-se em análise de metadados e balística.
\\[6pt]
Ambas têm um posicionamento de mercado diferente do proposto pelos autores.
\\[6pt]
Em suma, o plano de negócios citado tem características de inovação, tanto do ponto de vista mercadológico quanto do tecnológico, já que os autores se preocuparam em melhorar técnicas já existentes com o objetivo de ter um diferencial competitivo.
\\[6pt]
Logo, as práticas já existentes na área podem ser impactadas com as mudanças indicadas, principalmente, pelo fato da plataforma web ser interativa e capaz de dar resultados mais analíticos do que as demais avaliadas como \textit{benchmark}.
\\[6pt]
Os resultados obtidos não são de todo genéricos, já que é possível, ainda assim, alterar uma imagem sem que o método proposto pelo ELA detecte qualquer alteração.
\\[6pt]
Como trabalho futuro, o autor cita que novas ferramentas para auxiliar os usuários a obter uma melhor interpretação do resultado do ELA devem ser criadas e, também, extrair outros dados numéricos que possam ser incluídos como atributos de entrada para os classificadores.
\end{enumerate}



\item \textbf{Crítica}
    \begin{enumerate}[label*=\arabic*.]
      \item \textbf{Capacidade de inovação}
\\[6pt]      
Com exceção do método composto pelo ELA com filtro gaussiano, todos os demais são apenas repetições do estado da arte em análise de imagens.
\\[6pt]
Não há preocupação em relação a comparação dos métodos com outros existentes, já que eles estão sendo utilizados como experimentos para compor as técnicas que estarão disponíveis na plataforma web posteriormente. Isso fica ainda mais evidente na seção \textit{Intellectual Property}, onde os autores chegam à conclusão de que não é possível registrar patente da plataforma web já que ela não atende a dois requisitos das leis dos Estados Unidos para tal, que são:
\\[6pt]
1. A reprodução não pode ser óbvia, o que não ocorre, já que os próprios autores têm receio de serem copiados, uma vez que eles utilizam uma combinação de técnicas que estão disponíveis em outros artigos que estão publicados abertamente.
\\[6pt]
2. Inovação que, pelas mesmas leis, a invenção reivindicada deve ter sido antes de ser patenteada, descrita em uma publicação impressa, ou em uso público, em promoção, ou de outro modo disponível para o público, o que também não ocorre neste caso e, por consequência, dificulta o processo de obtenção de patente.
\\[6pt]
\item \textbf{Ausência de embasamento teórico}

Nenhum algoritmo foi mostrado, poucos conceitos matemáticos, nenhuma prova, o que não garante que o método é totalmente funcional. É difícil até mesmo avaliar qualquer tipo de erro conceitual, uma vez que maiores detalhes não foram fornecidos, acredito que de forma propositada, para evitar que detalhes do método fossem publicados inadvertidamente, oferecendo risco ao negócio como um todo.
\\[6pt]
Outro ponto já observado, o experimento com o \textit{dataset} \textit{Benchmark Dataset} não foi bem sucedido, conforme citado pelos autores, mas nenhuma métrica ou medida de erro foi detalhada.
\end{enumerate}


\item \textbf{Sugestões}
    \begin{enumerate}[label*=\arabic*.]
\item \textbf{Mudança de paradigma para \textit{open source}}

Uma possível mudança de paradigma seria abrir o código para a comunidade científica para obter sugestões, críticas, elogios e possíveis novas funcionalidades para a plataforma, de forma que não fique limitada a métodos já existentes ou que possam ser facilmente copiados. Com isso, as técnicas apresentadas podem ser mais bem elaboradas e expostas, sem qualquer tipo de problema em relação a segredo industrial ou patente envolvida.
\\[6pt]
Com este mesmo viés, a contribuição de outros grupos de pesquisa podem incluir embasamento teórico e fundamentado para este e novos métodos propostos.
    \end{enumerate}
\end{enumerate}

\clearpage

\item Implemente o método ELA (Error Level Analysis) em Python (apresente o 
\textbf{algoritmo} na prova e anexe o código em Python no arquivo zip).

Abaixo o algoritmo que foi implementado onde utilizei o pacote $pillow$ do python para as operações na imagem.
O código completo está em \textbf{source/ela.py}.

%Código
\begin{algorithm}[H]
    \SetKwData{Img}{ImagemOriginal}
    \SetKwData{ResavedImg}{ImagemComprimida}
    \SetKwData{ElaImg}{ImagemELA}
    \SetKwData{ImgPath}{Caminho\quad da\quad imagem}
    \SetKwData{ResavedPath}{Caminho\quad da\quad imagem comprimida}
    \SetKwFunction{abrirImagem}{abrirImagem}
    \SetKwFunction{salvarImagem}{salvarImagem}
    
    \SetAlgoLined
    \Entrada{Caminho\quad da\quad imagem, escala}
    \Saida{Imagem com ELA calculado}
    \BlankLine
    \Img $\leftarrow$ \abrirImagem{\ImgPath}\\
    \Se{\Img não é JPEG}{\Retorna{Nulo}}
    \BlankLine
    \salvarImagem{\ResavedPath, JPEG, qualidade}\\
    \ResavedImg $\leftarrow$ \abrirImagem{\ResavedPath}\\
    \BlankLine
    \Para{$x\leftarrow 0$ \Ate $largura\quad \Img$}{
        \Para{$y\leftarrow 0$ \Ate $altura\quad \Img$}{
            $pixel\_img\_original = \Img[x,y]$\\
            $pixel\_img\_comprimida = \ResavedImg[x,y]$\\
            $R = abs(pixel\_img\_original[0] - pixel\_img\_comprimida[0]) * escala$\\
            $G = abs(pixel\_img\_original[1] - pixel\_img\_comprimida[1]) * escala$\\
            $B = abs(pixel\_img\_original[2] - pixel\_img\_comprimida[2]) * escala$\\
            $\ElaImg[x,y] = [R, G, B]$
        }
    }
    \BlankLine
    \Retorna{\ElaImg}
    \caption{\textsc{Error Level Analysis}}
    \label{alg1}
\end{algorithm}

\item Teste seu algoritmo com as imagens que deixei no paca para este exercício. 
Quantas imagens seriam consideradas modificadas por esse método? Comente o resultado,
comparando com a sua intuição.

Uma vez submetida a imagem ao \textit{Error Level Analysis}, pode-se perceber pelo resultado que 
a região manipulada terá um nível de erro diferente das regiões não manipuladas. Logo, 
o nível de erro irá expor a região manipulada rotulando as regiões com maior alteração após
a imagem ser salva com um nível de qualidade inferior \cite{krawetz}.

Na Figura \ref{fig1}, podemos ver o resultado do ELA na imagem dada. As regiões com maior chance de ter
alterações são as que apresentam os pixels com maior brilho, uma vez que a alteração da imagem 
causa instabilidade nestas áreas.
\begin{figure}[H]
\centering
\begin{minipage}[b]{0.45\textwidth}
	\centering
        \includegraphics[scale=0.3]{Q3Images/cows_on_beach.jpg}
	\centerline{\small (1a) Imagem original}
\end{minipage}
\begin{minipage}[b]{0.45\textwidth}
	\centering
        \includegraphics[scale=0.3]{Q3Images/cows_on_beach_ela.jpg} 
	\centerline{\small (1b) Imagem com o resultado do ELA}
\end{minipage}
\caption{Aplicação do método ELA}
\label{fig1}
\end{figure}

Os resultados foram avaliados de acordo com o brilho das bordas que devem ser semelhantes no resultado 
da aplicação do ELA na imagem. Além disso, regiões de cores e texturas semelhantes na imagem
original, independentemente da cor, também devem ter cores aproximadamente similares no ELA \cite{berkeleywebsite}.

Isto posto, considerei que um total de 23 imagens foram alteradas de acordo com o método e avaliação a posteriori. 
A Tabela \ref{table-hoax-imgs} lista a avaliação do método nas imagens dadas. As imagens originais, bem como as
alteradas pelo método podem ser vistas no diretório \textbf{source/HoaxImages}.
\begin{table}[]
\centering
\caption{Tabela com o resultado da aplicação do método ELA}
\label{table-hoax-imgs}
\begin{tabular}{|l|l|} \hline
Imagem                 & Alterada? \\ \hline
2000\_snowballcat.jpg  & Y         \\
bikefail.jpg           & N         \\
blacklion01.jpg        & Y         \\
bouncing\_baby.jpg     & N         \\
broken\_road.jpg       & N         \\
businahole.jpg         & N         \\
cows\_on\_beach.jpg    & Y         \\
cursor.jpg             & N         \\
daliatom.jpg           & Y         \\
dononwater.jpg         & N         \\
eagles.jpg             & N         \\
frozenvenice.jpg       & Y         \\
glass\_butterfly.jpg   & Y         \\
goldfish\_hitler.jpg   & N         \\
hatfield.jpg           & N         \\
hellephant.jpg         & N         \\
hitlerbaby.jpg         & N         \\
horseinahole01.jpg     & N         \\
houseboat01.jpg        & Y         \\
iceberg.jpg            & N         \\
jumping\_giraffe.jpg   & Y         \\
kissing.jpg            & Y         \\
koala01.jpg            & N         \\
leap.jpg               & Y         \\
magic\_tap.jpg         & Y         \\
manitoba\_security.jpg & Y         \\
moonmelon01.jpg        & Y         \\
nikolatesla.jpg        & Y         \\
queensguard.jpg        & N         \\
rainbow\_tornado.jpg   & N         \\
rocket\_bike.jpg       & Y         \\
sharkswim.jpg          & Y         \\
shark\_roof.jpg        & N         \\
skiing\_egypt.jpg      & Y         \\
skullrose.jpg          & N         \\
spacechair.jpg         & Y         \\
tandembike.jpg         & N         \\
tattooguy01.jpg        & y         \\
tennis.jpg             & Y         \\
tentacle\_bldg.jpg     & Y         \\
trafficlights.jpg      & N         \\
tunnelface01.jpg       & N         \\
verydeep.jpg           & N         \\
vuitton.jpg            & Y         \\
whale.jpg              & N         \\
wienerplane.jpg        & Y         \\ \hline
\end{tabular}
\end{table}

\end{itemize}
%
%
%
\item[{\bf Q2.}] Esta questão refere-se à transformada de Fourier.
\begin{itemize}
\item Encontre a transformada de Fourier da função:
\begin{eqnarray*}
f(x) = \left\{ \begin{array}{rl} 
 7 &\mbox{ if $-5 < x < 5$} \\
 0 &\mbox{ caso contrário}
       \end{array} \right.
\end{eqnarray*}

Por definição, temos que a transformada de Fourier de um pulso
retangular de duração $D$ e altura $A$ tem a forma dada por:

\begin{eqnarray*}
    F(\omega) = ADsinc\bigg(\frac{\omega D}{2}\bigg)  = AD\frac{sin(\frac{\omega D}{2})}{\frac{\omega D}{2}}
\end{eqnarray*}

A função $f(x)$ pode ser representada graficamente como (Figura \ref{fig:fig2}):
\begin{figure}[H]
    \centering
    \includegraphics[width=0.3\textwidth]{Q3Images/pulse_function.png}
    \caption{Função pulso retangular}
    \label{fig:fig2}
\end{figure}

Onde:
\begin{eqnarray*}
f(x) = \left\{ \begin{array}{rl} 
 A, &\mbox{ $x \in  [\frac{-D}{2}, \frac{D}{2}]$} \\
 0, &\mbox{ $x \notin [\frac{-D}{2}, \frac{D}{2}]$}
       \end{array} \right.
\end{eqnarray*}

Logo, temos que A = 7 e D = 10 e, portanto, a transformada de Fourier da função $f(x)$ é:
\begin{align*}
    F(\omega) &= AD\frac{sin(\frac{\omega D}{2})}{\frac{\omega D}{2}} \\
              &= 7*10\frac{sin(\frac{\omega 10}{2})}{\frac{\omega 10}{2}} \\
              &= 70\frac{sin(\frac{\omega 10}{2})}{\frac{\omega 10}{2}} \\
              &= 70\frac{sin(5\omega)}{5\omega}
\end{align*}

%%%
\item Encontre a transformada de Fourier da função $ g(x) = f(x)\cos
   \omega_0 x$, sabendo que a transformada de Fourier de $f(x)$ é dada
   por $F(\omega)$

Tomando a propriedade da modulação:
\begin{align*}
    \mathcal{F}[x(t)cos(\omega_0t)] = \frac{1}{2}[F(\omega + \omega_0) + F(\omega - \omega_0)]
\end{align*}
Temos que a transformada de Fourier da função $g(x)$ é:
\begin{align*}
    G(\omega) = \frac{1}{2}F(\omega + \omega_0) + \frac{1}{2}F(\omega - \omega_0)
\end{align*}

%%%
\item Ache a inversa da transformada de Fourier de $G(\omega) =
  20\frac{\sin 5\omega}{5\omega}e^{-3\omega i}$

Por ora, ignorando a exponencial complexa de $G(\omega)$, podemos obter os valores de $A$ e $D$:
\begin{align*}
    20\frac{sin(5\omega)}{5\omega} = AD\frac{sin(\frac{\omega D}{2})}{\frac{\omega D}{2}} \\
    \\ 5\omega = \frac{\omega D}{2} \\
    10\omega = \omega D \\
    D = \frac{10\omega}{\omega} = 10 \\
    \\ AD = 20 \\
    A10 = 20 \\
    A = 2
\end{align*}

Tomando a representação da função $f(x)$ do pulso retangular de duração $D$ e amplitude $A$:
\begin{align*}
f(x) = A . rect(x) = \left\{ \begin{array}{rl}
 A, &\mbox{ $x \in  [\frac{-D}{2}, \frac{D}{2}]$} \\
 0, &\mbox{ $x \notin [\frac{-D}{2}, \frac{D}{2}]$}
       \end{array} \right.
\end{align*}

Vimos no primeiro item do exercício 2 que:
\begin{align*}
 A . rect\bigg(\frac{x}{D}\bigg) \xrightarrow{\mathscr{F}} ADsinc \bigg(\frac{\omega D}{2}\bigg)
\end{align*}

O que nos dá a forma do pulso retangular:
\begin{align*}
f(x) = 2 . rect\bigg(\frac{x}{10}\bigg) = \left\{ \begin{array}{rl} 
 2, &\mbox{ $x \in  [-5, 5]$} \\
 0, &\mbox{ $x \notin [-5, 5]$}
       \end{array} \right.
\end{align*}

Cuja representação gráfica é (Figura \ref{fig:fig3}):
\begin{figure}[H]
    \centering
    \includegraphics[width=0.3\textwidth]{Q3Images/pulse_function_2.png}
    \caption{Função pulso retangular com D = 10 e A = 2}
    \label{fig:fig3}
\end{figure}

Considerando agora a exponencial complexa, sabemos que ela representa um deslocamento no tempo,
que é $3$ neste caso e, portanto:
\begin{align*}
    2 . rect\bigg(\frac{x-3}{10}\bigg) \xrightarrow{\mathscr{F}} 20\frac{\sin 5\omega}{5\omega}e^{-3\omega i} \\
    g(x) = 2 . rect\bigg(\frac{x-3}{10}\bigg) = \left\{ \begin{array}{rl} 
     2, &\mbox{ $-2 < x < 8$} \\
     0, &\mbox{ caso contrário}
           \end{array} \right.
\end{align*}

\begin{figure}[H]
    \centering
    \includegraphics[width=0.3\textwidth]{Q3Images/pulse_function_3.png}
    \caption{Função pulso retangular com deslocamento no tempo}
    \label{fig:fig4}
\end{figure}

%%%
\item Calcule a DFT do sinal $f = \{1,3,5,3,1\}$

A Transformada Discreta de Fourier (DFT) provém da Série Discreta de Fourier (DFS) e é dada por \cite{broughton2009discrete}:
\begin{align*}
    X_k = \sum\limits_{m=0}^{N-1} x_m e^{\frac{-2\pi ikm}{N}},\qquad para\quad k=0,1,2,...,N-1
\end{align*}

Para realizar os cálculos devemos utilizar a identidade de Euler:
\begin{align*}
    e^{-i \pi} = \cos \pi - i\sin\pi
\end{align*}

Temos que N=5, ou seja, o tamanho do vetor do sinal dado por $f$, logo:
\begin{align*}
    &X_0 = (1e^0 + 3e^0 + 5e^0 + 3e^0 + 1e^0) &\\
    &X_1 = (1e^0 + 3e^{\frac{-2\pi i1}{5}} + 5e^{\frac{-2\pi i2}{5}} + 3e^{\frac{-2\pi i3}{5}} + 1e^{\frac{-2\pi i4}{5})} &\\
    &X_2 = (1e^0 + 3e^{\frac{-2\pi i2}{5}} + 5e^{\frac{-2\pi i4}{5}} + 3e^{\frac{-2\pi i6}{5}} + 1e^{\frac{-2\pi i8}{5})} &\\
    &X_3 = (1e^0 + 3e^{\frac{-2\pi i3}{5}} + 5e^{\frac{-2\pi i6}{5}} + 3e^{\frac{-2\pi i9}{5}} + 1e^{\frac{-2\pi i12}{5})} &\\
    &X_4 = (1e^0 + 3e^{\frac{-2\pi i4}{5}} + 5e^{\frac{-2\pi i8}{5}} + 3e^{\frac{-2\pi i12}{5}} + 1e^{\frac{-2\pi i16}{5})} &\\
    \\&X_0 = (1 + 3 + 5 + 3 + 1) &\\
    &X_1 = (1 + 3e^{\frac{-2\pi i}{5}} + 5e^{\frac{-4\pi i}{5}}  + 3e^{\frac{-6\pi i}{5}}  + 1e^{\frac{-8\pi i}{5})} &\\
    &X_2 = (1 + 3e^{\frac{-4\pi i}{5}} + 5e^{\frac{-8\pi i}{5}}  + 3e^{\frac{-12\pi i}{5}} + 1e^{\frac{-16\pi i}{5})} &\\
    &X_3 = (1 + 3e^{\frac{-6\pi i}{5}} + 5e^{\frac{-12\pi i}{5}} + 3e^{\frac{-18\pi i}{5}} + 1e^{\frac{-24\pi i}{5})} &\\
    &X_4 = (1 + 3e^{\frac{-8\pi i}{5}} + 5e^{\frac{-16\pi i}{5}} + 3e^{\frac{-24\pi i}{5}} + 1e^{\frac{-32\pi i}{5})}&
\end{align*}
Calculando cada exponencial complexa com a relação de Euler e substituindo os resultados na equação acima:
\begin{align*}
    &e^{\frac{-2\pi i}{5}} = cos(\frac{2\pi}{5}) - isen(\frac{2\pi}{5}) = 0.30902  - 0.95106i &\\
    &e^{\frac{-4\pi i}{5}} = cos(\frac{4\pi}{5}) - isen(\frac{4\pi}{5}) = -0.80902 - 0.58779i &\\
    &e^{\frac{-6\pi i}{5}} = cos(\frac{6\pi}{5}) - isen(\frac{6\pi}{5}) = -0.80902 + 0.58779i &\\
    &e^{\frac{-8\pi i}{5}} = cos(\frac{8\pi}{5}) - isen(\frac{8\pi}{5}) = 0.30902  - 0.95106i &\\
    &e^{\frac{-12\pi i}{5}} = cos(\frac{12\pi}{5}) - isen(\frac{12\pi}{5}) = 0.30902	-0.95106i &\\
    &e^{\frac{-16\pi i}{5}} = cos(\frac{16\pi}{5}) - isen(\frac{16\pi}{5}) = -0.80902 + 0.58779i &\\
    &e^{\frac{-18\pi i}{5}} = cos(\frac{18\pi}{5}) - isen(\frac{18\pi}{5}) = 0.30902 + 0.95106i &\\
    &e^{\frac{-24\pi i}{5}} = cos(\frac{24\pi}{5}) - isen(\frac{24\pi}{5}) = -0.80902 - 0.58779i &\\
    &e^{\frac{-32\pi i}{5}} = cos(\frac{32\pi}{5}) - isen(\frac{32\pi}{5}) = 0.30902	- 0.95106i&
\end{align*}

Temos então que o resultado da DFT é:
\begin{multline*}
    X[x] = 13.0, -4.236067-3.077683i, 0.236067+0.726542i, \\
           0.236067-0.726542i, -4.236067+3.077683i
\end{multline*}

\end{itemize}
%
%
%
\item[{\bf Q3.}]
\begin{itemize}
\item Calcule (apresente os cálculos) dos descritores de Fourier das
  figuras 5a e 5b. Lembre-se que os pontos da
  borda do quadrado serão representados por pontos no plano de
  Argand-Gauss. Isto é, cada ponto no plano passa a ser um número
  complexo e a borda passa a ser um vetor de pontos complexos, como
  num sinal, mas com valores complexos.

\begin{figure}[htb]
    \centering
    \begin{minipage}[b]{0.45\textwidth}
    	\centering
            \includegraphics[scale=0.2]{Q3Images/square1.jpg} 
    	\centerline{\small (5a) Quadrado de lado 1}
    \end{minipage}
    \begin{minipage}[b]{0.45\textwidth}
    	\centering
            \includegraphics[scale=0.3]{Q3Images/square3.jpg} 
    	\centerline{\small (5b) Quadrado de lado 3}
    \end{minipage}
\caption{Figuras no plano cartesiano}
\label{fig:fig5}
\end{figure}

\textbf{Resultados para a primeira imagem}

Para que possamos calcular os descritores de Fourier, devemos representar as coordenadas do quadrado (1,1), (-1,1), (-1,-1) e (1,-1), como 
coordenadas no plano de Argand-Gauss que, neste caso, são: (1 + i, -1 + i, -1 - i, 1 - i).

Os descritores de Fourier podem ser calculados a partir da DFT \cite{broughton2009discrete}:
\begin{align*}
    X_k = \sum\limits_{m=0}^{N-1} x_m e^{\frac{-2\pi ikm}{N}},\qquad para\quad k=0,1,2,...,N-1
\end{align*}

Da mesma forma que no item anterior, para realizar os cálculos devemos utilizar a identidade de Euler:
\begin{align*}
    e^{-i \pi} = \cos \pi - i\sin\pi
\end{align*}
	  
Temos que N=4, ou seja, o número de pontos no plano de Argand-Gauss, logo:
\begin{align*}
    &X_0 = (1 + i)e^0 + (-1 + i)e^0 + (-1 - i)e^0 + (1 - i)e^0 &\\
    &X_1 = (1 + i)e^0 + (-1 + i)e^{\frac{-2\pi i1}{4}} + (-1 - i)e^{\frac{-2\pi i2}{4}} + (1 - i)e^{\frac{-2\pi i3}{4}} &\\
    &X_2 = (1 + i)e^0 + (-1 + i)e^{\frac{-2\pi i2}{4}} + (-1 - i)e^{\frac{-2\pi i4}{4}} + (1 - i)e^{\frac{-2\pi i6}{4}} &\\
    &X_3 = (1 + i)e^0 + (-1 + i)e^{\frac{-2\pi i3}{4}} + (-1 - i)e^{\frac{-2\pi i6}{4}} + (1 - i)e^{\frac{-2\pi i9}{4}} &\\
    \\
    &X_0 = (1 + i) + (-1 + i) + (-1 - i) + (1 - i)) &\\
    &X_1 = (1 + i) + (-1 + i)e^{\frac{-2\pi i}{4}} + (-1 - i)e^{\frac{-4\pi i}{4}} + (1 - i)e^{\frac{-6\pi i}{4}} &\\
    &X_2 = (1 + i) + (-1 + i)e^{\frac{-4\pi i}{4}} + (-1 - i)e^{\frac{-8\pi i}{4}} + (1 - i)e^{\frac{-12\pi i}{4}} &\\
    &X_3 = (1 + i) + (-1 + i)e^{\frac{-6\pi i}{4}} + (-1 - i)e^{\frac{-12\pi i}{4}} + (1 - i)e^{\frac{-18\pi i}{4}} &
\end{align*}

Calculando cada exponencial complexa com a relação de Euler e substituindo os resultados na equação acima:
\begin{align*}
    &e^{\frac{-2\pi i}{4}} = cos(\frac{2\pi}{4}) - isen(\frac{2\pi}{4}) = -i &\\
    &e^{\frac{-4\pi i}{4}} = cos(\frac{4\pi}{4}) - isen(\frac{4\pi}{4}) = -1 &\\
    &e^{\frac{-6\pi i}{4}} = cos(\frac{6\pi}{4}) - isen(\frac{6\pi}{4}) = i &\\
    &e^{\frac{-8\pi i}{4}} = cos(\frac{8\pi}{4}) - isen(\frac{8\pi}{4}) = 1 &\\
    &e^{\frac{-12\pi i}{4}} = cos(\frac{12\pi}{4}) - isen(\frac{12\pi}{4}) = -1 &\\
    &e^{\frac{-18\pi i}{4}} = cos(\frac{18\pi}{4}) - isen(\frac{18\pi}{4}) = -i &
\end{align*}
Temos então que o resultado dos descritores de Fourier da imagem é:
\begin{align*}
    &X_0 = 0 &\\
    &X_1 = (1 + i) + (-1 + i)*(-i) + (-1 - i)*(-1) + (1 - i)*(i) &\\
    &X_2 = (1 + i) + (-1 + i)*(-1) + (-1 - i)*(1) + (1 - i)*(-1) &\\
    &X_3 = (1 + i) + (-1 + i)*(i) + (-1 - i)*(-1) + (1 - i)*(-i) &\\
    \\
    &X[x] = 0.0, 4 + 4i, 0.0, 0.0 &
\end{align*}

\textbf{Resultados para a segunda imagem}

Assim como fizemos no item anterior, devemos calcular as coordenadas da imagem para encontrarmos os descritores de Fourier.
Sabemos a coordenada x dos pontos inferior e superior do quadrilátero, que é 5 e, também, conhecemos a coordenada do eixo y para os ponto mais à esquerda e à direita que é igual a 4.

Para as demais coordenadas, basta calcular o valor da diagonal do quadrilátero que, neste caso, é a hipotenusa dos dois triângulos retângulos formados ao se dividir o quadrilátero ao meio, cujos lados possuem comprimento 3. 
    
Com base no teorema de Pitágoras temos que a hipotenusa c desse triângulo é dado por:
\begin{align*}
 c^2 &= a^2 + b^2 \\
 c^2 &= 3^2 + 3^2 \\
 c &= \sqrt{18}	     
\end{align*}

Consequentemente, temos que a diagonal é $d = \sqrt{18} = 4.24264$ e as coordenadas dos pontos da imagem, que denominei de $f$, são:
\begin{align*}
    f &= (5, 4 + \frac{4.24264}{2}), (5 - \frac{4.24264}{2}, 4), (5, 4 - \frac{4.24264}{2}), (5 + \frac{4.24264}{2}, 4) \\
    f &= (5, 6.12132), (2.87868, 4), (5, 1.87868), (7.12132, 4)
\end{align*}
    
Agora, devemos representar as coordenadas por pontos no Plano de Argand-Gauss, que são:
\begin{align*}
    f &= (5 + 6.12132i, 2.87868 + 4i, 5 + 1.87868i, 7.12132 + 4i)
\end{align*}

Os descritores de Fourier podem ser calculados a partir da DFT \cite{broughton2009discrete}:
\begin{align*}
    X_k = \sum\limits_{m=0}^{N-1} x_m e^{\frac{-2\pi ikm}{N}},\qquad para\quad k=0,1,2,...,N-1
\end{align*}

Novamente, utilizando a identidade de Euler:
\begin{align*}
    e^{-i \pi} = \cos \pi - i\sin\pi
\end{align*}
	  
Temos que N=4, ou seja, o número de pontos no plano de Argand-Gauss, logo:
\begin{align*}
    &X_0 = (5 + 6.12132i)e^0 + (2.87868 + 4i)e^0 + (5 + 1.87868i)e^0 + (7.12132 + 4i)e^0 &\\
    &X_1 = (5 + 6.12132i)e^0 + (2.87868 + 4i)e^{\frac{-2\pi i1}{4}} + (5 + 1.87868i)e^{\frac{-2\pi i2}{4}} + (7.12132 + 4i)e^{\frac{-2\pi i3}{4}} &\\
    &X_2 = (5 + 6.12132i)e^0 + (2.87868 + 4i)e^{\frac{-2\pi i2}{4}} + (5 + 1.87868i)e^{\frac{-2\pi i4}{4}} + (7.12132 + 4i)e^{\frac{-2\pi i6}{4}} &\\
    &X_3 = (5 + 6.12132i)e^0 + (2.87868 + 4i)e^{\frac{-2\pi i3}{4}} + (5 + 1.87868i)e^{\frac{-2\pi i6}{4}} + (7.12132 + 4i)e^{\frac{-2\pi i9}{4}} &\\
    \\
    &X_0 = (5 + 6.12132i) + (2.87868 + 4i) + (5 + 1.87868i) + (7.12132 + 4i)) &\\
    &X_1 = (5 + 6.12132i) + (2.87868 + 4i)e^{\frac{-2\pi i}{4}} + (5 + 1.87868i)e^{\frac{-4\pi i}{4}} + (7.12132 + 4i)e^{\frac{-6\pi i}{4}} &\\
    &X_2 = (5 + 6.12132i) + (2.87868 + 4i)e^{\frac{-4\pi i}{4}} + (5 + 1.87868i)e^{\frac{-8\pi i}{4}} + (7.12132 + 4i)e^{\frac{-12\pi i}{4}} &\\
    &X_3 = (5 + 6.12132i) + (2.87868 + 4i)e^{\frac{-6\pi i}{4}} + (5 + 1.87868i)e^{\frac{-12\pi i}{4}} + (7.12132 + 4i)e^{\frac{-18\pi i}{4}} &
\end{align*}

Calculando cada exponencial complexa com a relação de Euler e substituindo os resultados na equação acima:
\begin{align*}
    &e^{\frac{-2\pi i}{4}} = cos(\frac{2\pi}{4}) - isen(\frac{2\pi}{4}) = -i &\\
    &e^{\frac{-4\pi i}{4}} = cos(\frac{4\pi}{4}) - isen(\frac{4\pi}{4}) = -1 &\\
    &e^{\frac{-6\pi i}{4}} = cos(\frac{6\pi}{4}) - isen(\frac{6\pi}{4}) = i &\\
    &e^{\frac{-8\pi i}{4}} = cos(\frac{8\pi}{4}) - isen(\frac{8\pi}{4}) = 1 &\\
    &e^{\frac{-12\pi i}{4}} = cos(\frac{12\pi}{4}) - isen(\frac{12\pi}{4}) = -1 &\\
    &e^{\frac{-18\pi i}{4}} = cos(\frac{18\pi}{4}) - isen(\frac{18\pi}{4}) = -i &
\end{align*}
Temos então que o resultado dos descritores de Fourier da imagem é:
\begin{align*}
    &X_0 = (5 + 6.12132i) + (2.87868 + 4i) + (5 + 1.87868i) + (7.12132 + 4i) &\\
    &X_1 = (5 + 6.12132i) + (2.87868 + 4i)*(-i) + (5 + 1.87868i)*(-1) + (7.12132 + 4i)*(i) &\\
    &X_2 = (5 + 6.12132i) + (2.87868 + 4i)*(-1) + (5 + 1.87868i)*(1) + (7.12132 + 4i)*(-1) &\\
    &X_3 = (5 + 6.12132i) + (2.87868 + 4i)*(i) + (5 + 1.87868i)(-1) + (7.12132 + 4i)*(-i) &\\
    \\
    &X[x] = 20.0 + 16i,  8.428528i, 0.0, 0.0&
\end{align*}
     
\item Para confirmar que seus cálculos estão corretos, implemente um
  programa em Python que receba como entrada um vetor de números
  complexos (que são as coordenadas das bordas) e retorne os
  descritores de Fourier do vetor de entrada. Você pode usar as
  funções fornecidas pela biblioteca NUMPY para facilitar a
  programação.
  
O programa foi implementado em Python, com a biblioteca Numpy e está localizado em \textbf{source/fourier\_descriptors.py}.\\

\item Para reconstruir a curva, faça uma função que receba um vetor
  com os descritores de Fourier, um número $N$ de descritores a serem
  usados e grafique os pontos num plano cartesiano (para fazer a mesma 
figura que fizemos nos slides das aulas 15 e 16.

O programa foi implementado em Python, com a biblioteca Numpy e está localizado em \textbf{source/ifd.py}.\\
Basta executar o arquivo e passar um array com os descritores para que os pontos sejam desenhados. O tamanho
$N$ do vetor é calculado pelo programa.\\
A Figura \ref{fig:fig6}, por exemplo, foi gerada com o vetor de entrada:
\\$0+4j, 2+4j, 3+3j, 3+2j, 2+1j, 2-1j, 1-2j, 0-3j,-1-3j, -2-2j, -2+0j, -2+1j, -1+3j$

\begin{figure}[H]
    \centering
    \includegraphics[width=0.5\textwidth]{Q3Images/fdescriptors_points.png}
    \caption{Exemplo de pontos de contorno a partir de descritores de Fourier}
    \label{fig:fig6}
\end{figure}

\end{itemize}
%
%
%
\item[{\bf Q4.}] \textbf{Apenas para os alunos de pós-graduação}
\begin{itemize}
\item Leia o artigo do Torre e do Poggio
  \url{ftp://publications.ai.mit.edu/ai-publications/pdf/AIM-768.pdf}
  e faça um resumo de acordo com as indicações que deixei no
  paca (artigos sobre como fazer um resumo). 

  \begin{enumerate}
\item \textbf{Resumo}
\begin{enumerate}[label*=\arabic*.]
    \item \textbf{Motiva��o}
    TODO
    \\[6pt]

    \item \textbf{Contribui��o}
    TODO
    \\[6pt]

    \item \textbf{Metodologia}
    TODO
    \\[6pt]

    \item \textbf{Conclus�o}
    TODO
\end{enumerate}

\item \textbf{Cr�tica}
\begin{enumerate}[label*=\arabic*.]
    \item \textbf{Capacidade de inova��o}
    TODO
    \\[6pt]
    
    \item \textbf{Aus�ncia de embasamento te�rico}
    TODO
\end{enumerate}

\item \textbf{Sugest�es}
\begin{enumerate}[label*=\arabic*.]
    \item \textbf{Mudan�a de paradigma para \textit{open source}}
    TODO
\end{enumerate}
\end{enumerate}


\item O que é um problema mal-posto?
\\Um problema matemático é bem-posto, de acordo com a definição de Hadamard, se ele cumpre as seguintes condições:
\begin{enumerate}
  \item Existe solução;
  \item A solução é única;
  \item A solução tem uma dependência contínua com os dados de entrada (quando o problema não é somente bem-posto, 
  mas também, bem-condicionado, portanto, é robusto contra ruído);
\end{enumerate}  

Logo, um problema é dito mal-posto se alguma das condições acima não é satisfeita.  
\\No artigo em questão, o problema mal-posto trata-se da diferenciação numérica, que pode ser considerada como um problema
inverso da equação integral:

\begin{equation} \label{eq:inverse_int}
   g(x) = \int_{-\infty}^{x} f(\tilde{x}) d\tilde{x}
\end{equation}

onde $f(x)$ deve ser recuperada a partir do conhecimento dos dados $g(x)$ que, em geral, é obtido apenas em uma estrutura discreta.
\\

\item O que é regularização?
\\Regularização, em linhas gerais, é a aproximação de um problema mal-posto por uma família de problemas bem-postos \cite{Engl-1981}.
\\Conforme visto no item anterior, como temos um problema de diferenciação numérica e é sabido que ele é mal-posto, os autores 
avaliaram alguns métodos de regularização para, então, aproximar uma solução.
\\Um dos possíveis métodos são os operadores de Tikhonov que, para equações de convolução como em (\ref{eq:inverse_int}),
os operadores correspondem à convolução de $g(x)$ com um filtro $F(x, \alpha)$, onde $\alpha$ é um parâmetro.
\\

\item Qual a importância do teorema apresentado no artigo?
\textbf{TODO}

\item O que são filtros de banda limitada? Qual a sua importância no artigo?
\textbf{TODO}

\item Quais são os métodos de encontrar borda apresentados no artigo?
Os métodos de detecção de borda apresentados no artigo são:
\begin{enumerate}[label*=\arabic*.]
    \item \textbf{Difference of Boxes - DOB}\\
    Proposto por Herskovitz e Binford, o filtro DOB é do tipo suporte limitado e utiliza a função de Haar para filtragem direcional
    ou diferença de funções para filtragem rotacional.
    A função de Haar tem a propriedade de ser um filtro ótimo de suporte limitado que maximiza a relação sinal-ruído.
    \\
    \item \textbf{Shanmugam, Dickey and Green}\\
    O operador proposto por Shanmugam, Dickey e Green, como citado por Torre e Poggio, é um filtro ótimo na medida em que produz
    o máximo de energia dentro de um intervalo de um dado espaço ao redor da borda. Eles analisaram os casos contínuo e discreto e
    concluíram que a implementação pode ser realizada por meio da Transformada Rápida de Fourier (FFT).
    \\
    \item \textbf{Laplacian of Gaussian - LOG}\\
    Proposto por Marr e Hildreth, o Laplaciano do Gaussiano combina a filtragem Gaussiana para a suavização da imagem, com o operador Laplaciano
    que localiza as bordas pela presença de um cruzamento em zero na derivada segunda com um pico acentuado correspondente à derivada primeira \cite{pedrini2008analise}.\\
    A saída do operador é dada pela operação de convolução
    \begin{equation} \label{eq:log}
        (\nabla^2 G(x, y)) * f(x, y)
    \end{equation}
    onde $f(x, y)$ é uma imagem suavizada por uma função Gaussiana.
    \\
    \item \textbf{Haralick}\\
    O método proposto por Haralick consiste em marcar um pixel como parte integrante da borda se na sua vizinhança há passagens em zero da segunda 
    derivada ao longo do gradiente. Com o propósito de calcular as derivadas, Haralick aproxima e interpola os valores de intensidade amostrados
    com polinômios discretos de Chebychev.
    \\
    \item \textbf{Canny}\\
    De acordo com Pedrini \cite{pedrini2008analise}, com o intuito de otimizar a localização de pontos de borda na presença de ruído, o método de Canny suaviza a imagem por meio de um filtro Gaussiano e, em seguida, a magnitude e a direção do gradiente são calculadas utilizando aproximações baseadas em diferenças finitas para as derivadas parciais.\\
    Após o cálculo do gradiente, a borda é localizada utilizando supressão não-máxima, ou seja, apenas os pontos cuja magnitude seja localmente 
    máxima na direção do gradiente.\\
    Fragmentos espúrios causados pela presença de ruído ou textura fina são então removidos com a aplicação de dois limiares diferentes,
    cujo processo é conhecido como \textit{limiarização por histerese}.
    \\
\end{enumerate}

\end{itemize}
\end{itemize}
\clearpage

\printbibliography
\end{document}